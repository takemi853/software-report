\documentclass[12pt]{jsarticle}
\usepackage[dvipdfmx]{graphicx}
\usepackage{listings}
\lstset{
  basicstyle={\ttfamily},
  identifierstyle={\small},
  commentstyle={\smallitshape},
  keywordstyle={\small\bfseries},
  ndkeywordstyle={\small},
  stringstyle={\small\ttfamily},
  frame={tb},
  breaklines=true,
  columns=[l]{fullflexible},
  numbers=left,
  xrightmargin=0zw,
  xleftmargin=3zw,
  numberstyle={\scriptsize},
  stepnumber=1,
  numbersep=1zw,
  lineskip=-0.5ex
}               
\title{ソフトウェア工学 第1回レポート}
\author{竹味 和輝 31114078}
\date{2021年4月20日}
\begin{document}
\maketitle

出題授業日:4/15
\section{問題}
リスク管理における5種類の対処を退職リスクを例にして説明する。
 
\begin{itemize}
  \item 予防:退職をしないように事前に環境や給料など不満がないかを聞き対応を行う。
  \item 回避:退職をしないように説得したり、定年の制限などを見直したりする。
  \item 転嫁:抜けてしまった人の代わりに、その人の担当分を他の人に依頼する。
  \item 軽減:退職によって人が抜けることで担当者がいなくなるというリスクを軽減させるために、後継者を育てておく。
  \item 受容:1人抜けたことを受け入れ、今いるメンバーでできることを見直しながらプロジェクトを進める。
\end{itemize}  

今回、リスクマネジメントについて学習し課題に取り組んだが、リスクへの対応についてこの5つの分類のように、整理することができているとリスクが起きたときにも対応しやすいと感じられるので、実際に自分で活動するときにもこの考え方を持ちながら活動できたらと思った。
 

\end{document}
